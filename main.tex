\documentclass{article}
\usepackage[utf8]{inputenc}
\usepackage{kotex}
\usepackage{amsmath}
\usepackage{amssymb}
\usepackage{amsfonts}
\usepackage{graphicx}
\graphicspath{ {./images/} }
\usepackage{dcolumn}
\usepackage{bm}
\usepackage{hyperref}
\usepackage{mathptmx}
\usepackage{tikz}
\usetikzlibrary{positioning}
\usepackage{textcomp}
\usepackage{natbib}
\usepackage[rightcaption]{sidecap}
\usepackage{wrapfig}


\title{소쉬르}
\author{박석훈 / 학생 / 사회학과 ­}
\date{May 2022}

\begin{document}

\maketitle

\section{Introduction}



단어들이 관계를 맺고 있는 방식, 질서. 개별 단어가 중요하지 않음. 실체적 언어관에서는 단어 자체가 독립적인 의미를 지니고 있었다. 구조주의적 언어관에서는 개별 고정된 의미가 아니라, 단어들 간의 관계. 눈에 보이지 않는 그물. 유기적으로 촘촘하게 엮인 그 관계가 중요. 단어가 다른 단어와 맺고 있는 관계가 변하기 때문에 변한다. 개별적인 현상이 중요한 것이 아니라, 그것을 가능하게 하는 전체가 중요. \\

소쉬르의 체계 = 구조. 나무가 현실 속에 존재하고 우리가 그 단어를 붙인 것. 길이 있기 때문에 길이라는 단어를 고안해 놓은 것. 그런데 길이라는 단어가 없었을 때는? 전통적인 언어학자들은 전통적인 기호가 없더라도, 그 대상에 상응하는 개념을 이미 가지고 있지 않았을까?라는 의심을 함.\\

그런데 소쉬르는 그러한 개념따위는 없었다. 길이라는 단어가 없었다면, 길이라는 현상 자체를 경험하지 못했을 것. 내가 사용하는 언어 체계에 속하지 않은 사태에는 유의미한 경험을 하지 못할 것. 내 언어 구조 안에서만 의미있는 방식을 통해 의미있는 경험을 할 수 있음. 기의가 기표보다 먼저 존재하지 않음. \\

소쉬르 이전에는 의식이 언어를 만들어낸다고 생각. 물병이라고 이름을 붙여야지 라고 생각. 생각이 먼저 있고, 기호가 생성된다. 그런데 소쉬르는 뒤집어서 언어가 있기 때문에 의식이 있다고 생각. 언어 체계, 언어 구조가 있기 때문에 유의미한 경험을 하고 생각을 할 수 있다. 왜 언어가 먼저인가? 왜 언어는 단순히 도구에 불과한 것이 아닌가? 의미의 주인이 언어가 된 것. 즉 기표는 기의를 대신하는 도구적 지위를 갖는 것이 아니고, 기의는 기표보다 먼저 존재하지 않는다. 구조주의자들은 인식이 조건에 의해서 제약받는다라고 계속 주장.\\

시대나 문화에 따라서 동일한 경험이 다르게 느껴질 수 있다는 것. 화장실에 가는 것도 예전에는 푸세식. 그런데 지금은 비교적 편안하고 안락한 경험이 될 수 있음. 거기에다가 개개인별로도 차이가 있음. 물 한 병을 경험하더라도, 화장실에 가는 사건 하나를 경험하더라도 그 경험은 직접적인 것이 될 수 없음. 매개될 수밖에 없고, 인식의 조건으로서 구조를 매개로 해서 간접적으로만 가능. 철학적으로는 나는 직접적으로 실재에 접근하지 못하고, 나는 간접적으로만 구조를 통해서 실재에 접근하는 것.\\

언어에 없는 것은 존재론적으로 부재하다. 에스키모는 눈을 30가지가 넘게 표현하는데 한국인은 대여섯가지밖에 되지 않는다. 우리가 직접적으로 세계를, 현실을 경험하는 것이 아니라 구조를 매개를 통해서 간접적으로만 경험하고 세계를 그런식으로만 인식할 수 있다면, 우리가 구조에 속하는 것들만 그 구조안에서 의미가 부여된 것들만 가지고 그것을 받아들일 수 있다면, 자유롭게 받아들일 수 있는 데에 제한되어 있다면, 결국 구조주의적 관점에 근거할 때 의미의 주인은 이데아나 의식이 아닌 것. 객관주의적 인식론(플라톤의 이데아)는 형상들을 기준으로 해서 객관적으로 실재하는 보편자들을 기준으로 인식하는 것. 반면에 의식철학자(주관주의 철학자)는 의식의 기준이 인간의 안에 있음. 칸트적으로는 인식의 틀이 내 의식 안에 있기 때문에 나는 물병을 물병으로 인식할 수 있는 것. \\

물병이 구조 안에서 형성되기 전에도 그것을 볼 수 있고 경험할 수 있음. 다만 그것을 물병으로서 존재하는 것은 아니고, 지금 내가 경험한 것이 무엇인지를 모를 뿐. ₩₩

구조주의를 통해섭 보이지 않는 것을 보는 가능성을 줄 수 있다. 다빈치는 회화의 목적을 자연을 모방하는 것이라고 보았음. 자연을 잘 모방함으로써 얻는 그림은 무엇일까. 그것은 `환영'이다. 그것이 아닌 것을 보면서 그것이라고 착각하는 것이 환영. 그러한 환영에는 그것이 진짜라고 하는 믿음이 있고, 그 믿음은 기만(거짓말)의 결과이다. 내가 아무리 윤곽선을 잘 따냈다고 하더라도, 그것이 진짜 환영처럼 보여야하는데, 그 윤곽선 때문에 환영에 방해를 받을 수 있음. 자연에 없는 윤곽선이 있음으로써, 환영의 경험이 실패할 수 있음. 즉, 관람객을 환영에 빠트리기 위해서 개발하는 기법이 바로 `스포마토' 기법. 윤곽선은 없고, 소재는 살리면서도 경계와 조화가 되게끔 한 것이 다빈치.

하지만 스포마토 기법 그 자체도 보이면 안 된다. 즉, 스포마토 기법이 성공하기 위해서는 스포마토 그 자체도 보여서는 안 되는 것. 꽃병은 현실이지만 꽃병을 그린 그림은 비현실. 

Q. 관찰이 관찰을 하는 것과 스포마토는?

구조가 작동하는 방식이 스포마토 붓질이 작동하는 방식과 유사하다. 구조는 대개의 경우 포함과 배제라는 이분법적 방식을 통해서 작동. 순수한 것은 포함되는 반면, 불순한 것은 배제하는 것. 상징적 질서를 구축하는 데 사용되고, 상징적 질서의 구축 자체가 포함과 배제를 통해서 발생하는 것. 현실 속에서 사회는 그냥 있음. 상징적 질서가 구축되면서, 사회에 경계선이 그어짐. 순수한 것과 불순한 것 사이에는 넘을 수 없는 선이 있음. 현실 공간을 상징적 질서를 통해서 재단하는 것

Q. 하버마스의 생활세계의 식민화. 생활세계. 공론장.

구분되고 나면은 현실적으로 보이지 않는 경계선에 의해서 서로가 분리가 됨. 그렇게 분리가 되었기 때문에, 나쁜쪽은 좋은 쪽에서의 배제가 됨. 대개의 경우 닮은 이들은 우리라는 경계선 안으로 포섭이 되고, 다른 사람들은 바깥으로 추출이 됨. 하지만 배제는 우리 눈에 잘 보이지 않기도 함(벤치에서의 손잡이-노숙인). 보이지 않는 배제 `암묵적 배제' 

Q. 영화 더 헬프에서의 colored?

왜 우리는 배제를 보지 못할까? 배제를 보는 인식의 장애물이 있는가? 환영을 만들어내는 구조가 존재하는가? 배제를 인식하지 못하게 하는 작용이 있다는 것. 배제적 질서를 작동하게 하는 구조가 우리의 인식의 조건이기 때문은 아닌가? 구조가 우리가 보는 시각을, 판단능력을, 인식을 조건짓는것.
있는 그대로 보는 것이 아니라, 구조라는 색안경을 끼는 것?

Q. 수용체와 작동체를 중개하는 것이 인식? 

인식이 상대적이다.는 것은 인식이 선택적이라는 것이고, 살아가는 유기체가 살아가는 세계 역시도 상대적이다. 즉 서로 다른 세계를 갖는 것. 

시점이나 세계관, 관점이 바뀌면 세계가 변한다. 내가 경험하는 세계의 의미 자체가 변한다. - 양화사?

구조주의에서는 상징적 질서가 있고, 상징적 질서가 해당 구조에 대해서 작동할 때, 그 구조에 의해 자체적으로 배제된 것이 그 구조를 공유하고 있는 구성원들이 인식할 가능성이 없음. `인식적 배제'를 깨닫지 못함. 구조가 배제하는 것을 인식한다는 것은 대단히 어렵고, 사유 주제로 삼기도 어렵고 대단히 어려움. 그러한 암묵적 배제를 환영이라고 할 수 있을 것.

만약 우리가 보이지 않는 방식으로 사회적 약자를 배제한다면, 보이지 않는 배제, 암묵적 배제는 보이지 않기 때문에 가해자를 특정하기 어렵고 피해 그 자체를 정의하기도 어렵다. 사실상 배제의 질서가 작동하고 있다면, 그 구조를 공유하고 있는 우리는 어쩔 수 없이 인식론적이든 실천론적이든 배제를 할 수밖에 없음. 우리들 중 그 누구도 배제를 행한다는 자각을 할 수 없고, 우리들 중 누구도 보이지 않는 배제로부터 자유로울 수 없음. 배제가 저항의 대상이라면, 우리는 보이지 않는 것을 보아야 하는데. 

`보이지 않는 것을 본다'의 의미.



\end{document}
