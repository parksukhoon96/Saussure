\documentclass{article}
\usepackage[utf8]{inputenc}
\usepackage{kotex}
\usepackage{amsmath}
\usepackage{amssymb}
\usepackage{amsfonts}
\usepackage{graphicx}
\graphicspath{ {./images/} }
\usepackage{dcolumn}
\usepackage{bm}
\usepackage{hyperref}
\usepackage{mathptmx}
\usepackage{tikz}
\usetikzlibrary{positioning}
\usepackage{textcomp}
\usepackage{natbib}
\usepackage[rightcaption]{sidecap}
\usepackage{wrapfig}


\title{소쉬르}
\author{박석훈 / 학생 / 사회학과 ­}
\date{May 2022}

\begin{document}

\maketitle

\section{Introduction}



단어들이 관계를 맺고 있는 방식, 질서. 개별 단어가 중요하지 않음. 실체적 언어관에서는 단어 자체가 독립적인 의미를 지니고 있었다. 구조주의적 언어관에서는 개별 고정된 의미가 아니라, 단어들 간의 관계. 눈에 보이지 않는 그물. 유기적으로 촘촘하게 엮인 그 관계가 중요. 단어가 다른 단어와 맺고 있는 관계가 변하기 때문에 변한다. 개별적인 현상이 중요한 것이 아니라, 그것을 가능하게 하는 전체가 중요. \\

소쉬르의 체계 = 구조. 나무가 현실 속에 존재하고 우리가 그 단어를 붙인 것. 길이 있기 때문에 길이라는 단어를 고안해 놓은 것. 그런데 길이라는 단어가 없었을 때는? 전통적인 언어학자들은 전통적인 기호가 없더라도, 그 대상에 상응하는 개념을 이미 가지고 있지 않았을까?라는 의심을 함.\\

그런데 소쉬르는 그러한 개념따위는 없었다. 길이라는 단어가 없었다면, 길이라는 현상 자체를 경험하지 못했을 것. 내가 사용하는 언어 체계에 속하지 않은 사태에는 유의미한 경험을 하지 못할 것. 내 언어 구조 안에서만 의미있는 방식을 통해 의미있는 경험을 할 수 있음. 기의가 기표보다 먼저 존재하지 않음. \\

소쉬르 이전에는 의식이 언어를 만들어낸다고 생각. 물병이라고 이름을 붙여야지 라고 생각. 생각이 먼저 있고, 기호가 생성된다. 그런데 소쉬르는 뒤집어서 언어가 있기 때문에 의식이 있다고 생각. 언어 체계, 언어 구조가 있기 때문에 유의미한 경험을 하고 생각을 할 수 있다. 왜 언어가 먼저인가? 왜 언어는 단순히 도구에 불과한 것이 아닌가? 의미의 주인이 언어가 된 것. 즉 기표는 기의를 대신하는 도구적 지위를 갖는 것이 아니고, 기의는 기표보다 먼저 존재하지 않는다. 구조주의자들은 인식이 조건에 의해서 제약받는다라고 계속 주장.\\

시대나 문화에 따라서 동일한 경험이 다르게 느껴질 수 있다는 것. 화장실에 가는 것도 예전에는 푸세식. 그런데 지금은 비교적 편안하고 안락한 경험이 될 수 있음. 거기에다가 개개인별로도 차이가 있음. 물 한 병을 경험하더라도, 화장실에 가는 사건 하나를 경험하더라도 그 경험은 직접적인 것이 될 수 없음. 매개될 수밖에 없고, 인식의 조건으로서 구조를 매개로 해서 간접적으로만 가능. 철학적으로는 나는 직접적으로 실재에 접근하지 못하고, 나는 간접적으로만 구조를 통해서 실재에 접근하는 것.\\

언어에 없는 것은 존재론적으로 부재하다. 에스키모는 눈을 30가지가 넘게 표현하는데 한국인은 대여섯가지밖에 되지 않는다. 우리가 직접적으로 세계를, 현실을 경험하는 것이 아니라 구조를 매개를 통해서 간접적으로만 경험하고 세계를 그런식으로만 인식할 수 있다면, 우리가 구조에 속하는 것들만 그 구조안에서 의미가 부여된 것들만 가지고 그것을 받아들일 수 있다면, 자유롭게 받아들일 수 있는 데에 제한되어 있다면, 결국 구조주의적 관점에 근거할 때 의미의 주인은 이데아나 의식이 아닌 것. 객관주의적 인식론(플라톤의 이데아)는 형상들을 기준으로 해서 객관적으로 실재하는 보편자들을 기준으로 인식하는 것. 반면에 의식철학자(주관주의 철학자)는 의식의 기준이 인간의 안에 있음. 칸트적으로는 인식의 틀이 내 의식 안에 있기 때문에 나는 물병을 물병으로 인식할 수 있는 것. \\

물병이 구조 안에서 형성되기 전에도 그것을 볼 수 있고 경험할 수 있음. 다만 그것을 물병으로서 존재하는 것은 아니고, 지금 내가 경험한 것이 무엇인지를 모를 뿐. 

\end{document}
